\documentclass{ximera}
\title{Thevenin and Norton equivalent circuits}
\begin{document}
\begin{abstract}
This activity explains the theory of Thevenin and Norton equivalent circuits.
\end{abstract}

\maketitle

% Introduction: being able to simplify or black-box circuits is helpful

% Some linear circuits and their IV plots

\begin{exercise}
Write an equation for the current $i$ as a function of the voltage $V$.  Your solution should have the form $i = a \cdot V + b$, but a or b may be zero.

\begin{solution}
  \begin{hint}
  \begin{prompt}
    What is the y-intercept? \answer{0}
  \end{prompt}
  \end{hint}
  \begin{hint}
  Find the slope.  This will give you $a$.
  \end{hint}
  
  $i = $ \answer{0.001 V}
\end{solution}

Resistors have iV curves which are straight lines passing through the origin, and the slope is the inverse of the resistance.
\end{exercise}


Now let's figure out current sources.
\begin{exercise}
Write an equation for the current $i$ as a function of the voltage $V$.  Again, your solution should have the form $i = a \cdot V + b$, but a or b may be zero.

\begin{solution}
  \begin{hint}
    Remember that a current source outputs the same current regardless of the voltage across it.
  \end{hint}
  \begin{hint}
  \begin{prompt}
    What is the y-intercept? \answer{0}
  \end{prompt}
  \end{hint}
 
  $i = $ \answer{1}
\end{solution}

\end{exercise}

%Voltage sources are a little trickier:


% Every linear circuit produces a single line on the IV graph
% It makes sense, then, that we could represent every linear circuit using just a slope and an offset.

% What gives us slope?
% What gives us vertical offset?
% Horizontal offset?
% Why does the resistor have to be in series/parallel?

\end{document}
